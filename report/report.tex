\documentclass[12pt]{report}


%% ODER: format ==         = "\mathrel{==}"
%% ODER: format /=         = "\neq "
%
%
\makeatletter
\@ifundefined{lhs2tex.lhs2tex.sty.read}%
  {\@namedef{lhs2tex.lhs2tex.sty.read}{}%
   \newcommand\SkipToFmtEnd{}%
   \newcommand\EndFmtInput{}%
   \long\def\SkipToFmtEnd#1\EndFmtInput{}%
  }\SkipToFmtEnd

\newcommand\ReadOnlyOnce[1]{\@ifundefined{#1}{\@namedef{#1}{}}\SkipToFmtEnd}
\usepackage{amstext}
\usepackage{amssymb}
\usepackage{stmaryrd}
\DeclareFontFamily{OT1}{cmtex}{}
\DeclareFontShape{OT1}{cmtex}{m}{n}
  {<5><6><7><8>cmtex8
   <9>cmtex9
   <10><10.95><12><14.4><17.28><20.74><24.88>cmtex10}{}
\DeclareFontShape{OT1}{cmtex}{m}{it}
  {<-> ssub * cmtt/m/it}{}
\newcommand{\texfamily}{\fontfamily{cmtex}\selectfont}
\DeclareFontShape{OT1}{cmtt}{bx}{n}
  {<5><6><7><8>cmtt8
   <9>cmbtt9
   <10><10.95><12><14.4><17.28><20.74><24.88>cmbtt10}{}
\DeclareFontShape{OT1}{cmtex}{bx}{n}
  {<-> ssub * cmtt/bx/n}{}
\newcommand{\tex}[1]{\text{\texfamily#1}}	% NEU

\newcommand{\Sp}{\hskip.33334em\relax}


\newcommand{\Conid}[1]{\mathit{#1}}
\newcommand{\Varid}[1]{\mathit{#1}}
\newcommand{\anonymous}{\kern0.06em \vbox{\hrule\@width.5em}}
\newcommand{\plus}{\mathbin{+\!\!\!+}}
\newcommand{\bind}{\mathbin{>\!\!\!>\mkern-6.7mu=}}
\newcommand{\rbind}{\mathbin{=\mkern-6.7mu<\!\!\!<}}% suggested by Neil Mitchell
\newcommand{\sequ}{\mathbin{>\!\!\!>}}
\renewcommand{\leq}{\leqslant}
\renewcommand{\geq}{\geqslant}
\usepackage{polytable}

%mathindent has to be defined
\@ifundefined{mathindent}%
  {\newdimen\mathindent\mathindent\leftmargini}%
  {}%

\def\resethooks{%
  \global\let\SaveRestoreHook\empty
  \global\let\ColumnHook\empty}
\newcommand*{\savecolumns}[1][default]%
  {\g@addto@macro\SaveRestoreHook{\savecolumns[#1]}}
\newcommand*{\restorecolumns}[1][default]%
  {\g@addto@macro\SaveRestoreHook{\restorecolumns[#1]}}
\newcommand*{\aligncolumn}[2]%
  {\g@addto@macro\ColumnHook{\column{#1}{#2}}}

\resethooks

\newcommand{\onelinecommentchars}{\quad-{}- }
\newcommand{\commentbeginchars}{\enskip\{-}
\newcommand{\commentendchars}{-\}\enskip}

\newcommand{\visiblecomments}{%
  \let\onelinecomment=\onelinecommentchars
  \let\commentbegin=\commentbeginchars
  \let\commentend=\commentendchars}

\newcommand{\invisiblecomments}{%
  \let\onelinecomment=\empty
  \let\commentbegin=\empty
  \let\commentend=\empty}

\visiblecomments

\newlength{\blanklineskip}
\setlength{\blanklineskip}{0.66084ex}

\newcommand{\hsindent}[1]{\quad}% default is fixed indentation
\let\hspre\empty
\let\hspost\empty
\newcommand{\NB}{\textbf{NB}}
\newcommand{\Todo}[1]{$\langle$\textbf{To do:}~#1$\rangle$}

\EndFmtInput
\makeatother
%
%
%
%
%
%
% This package provides two environments suitable to take the place
% of hscode, called "plainhscode" and "arrayhscode". 
%
% The plain environment surrounds each code block by vertical space,
% and it uses \abovedisplayskip and \belowdisplayskip to get spacing
% similar to formulas. Note that if these dimensions are changed,
% the spacing around displayed math formulas changes as well.
% All code is indented using \leftskip.
%
% Changed 19.08.2004 to reflect changes in colorcode. Should work with
% CodeGroup.sty.
%
\ReadOnlyOnce{polycode.fmt}%
\makeatletter

\newcommand{\hsnewpar}[1]%
  {{\parskip=0pt\parindent=0pt\par\vskip #1\noindent}}

% can be used, for instance, to redefine the code size, by setting the
% command to \small or something alike
\newcommand{\hscodestyle}{}

% The command \sethscode can be used to switch the code formatting
% behaviour by mapping the hscode environment in the subst directive
% to a new LaTeX environment.

\newcommand{\sethscode}[1]%
  {\expandafter\let\expandafter\hscode\csname #1\endcsname
   \expandafter\let\expandafter\endhscode\csname end#1\endcsname}

% "compatibility" mode restores the non-polycode.fmt layout.

\newenvironment{compathscode}%
  {\par\noindent
   \advance\leftskip\mathindent
   \hscodestyle
   \let\\=\@normalcr
   \let\hspre\(\let\hspost\)%
   \pboxed}%
  {\endpboxed\)%
   \par\noindent
   \ignorespacesafterend}

\newcommand{\compaths}{\sethscode{compathscode}}

% "plain" mode is the proposed default.
% It should now work with \centering.
% This required some changes. The old version
% is still available for reference as oldplainhscode.

\newenvironment{plainhscode}%
  {\hsnewpar\abovedisplayskip
   \advance\leftskip\mathindent
   \hscodestyle
   \let\hspre\(\let\hspost\)%
   \pboxed}%
  {\endpboxed%
   \hsnewpar\belowdisplayskip
   \ignorespacesafterend}

\newenvironment{oldplainhscode}%
  {\hsnewpar\abovedisplayskip
   \advance\leftskip\mathindent
   \hscodestyle
   \let\\=\@normalcr
   \(\pboxed}%
  {\endpboxed\)%
   \hsnewpar\belowdisplayskip
   \ignorespacesafterend}

% Here, we make plainhscode the default environment.

\newcommand{\plainhs}{\sethscode{plainhscode}}
\newcommand{\oldplainhs}{\sethscode{oldplainhscode}}
\plainhs

% The arrayhscode is like plain, but makes use of polytable's
% parray environment which disallows page breaks in code blocks.

\newenvironment{arrayhscode}%
  {\hsnewpar\abovedisplayskip
   \advance\leftskip\mathindent
   \hscodestyle
   \let\\=\@normalcr
   \(\parray}%
  {\endparray\)%
   \hsnewpar\belowdisplayskip
   \ignorespacesafterend}

\newcommand{\arrayhs}{\sethscode{arrayhscode}}

% The mathhscode environment also makes use of polytable's parray 
% environment. It is supposed to be used only inside math mode 
% (I used it to typeset the type rules in my thesis).

\newenvironment{mathhscode}%
  {\parray}{\endparray}

\newcommand{\mathhs}{\sethscode{mathhscode}}

% texths is similar to mathhs, but works in text mode.

\newenvironment{texthscode}%
  {\(\parray}{\endparray\)}

\newcommand{\texths}{\sethscode{texthscode}}

% The framed environment places code in a framed box.

\def\codeframewidth{\arrayrulewidth}
\RequirePackage{calc}

\newenvironment{framedhscode}%
  {\parskip=\abovedisplayskip\par\noindent
   \hscodestyle
   \arrayrulewidth=\codeframewidth
   \tabular{@{}|p{\linewidth-2\arraycolsep-2\arrayrulewidth-2pt}|@{}}%
   \hline\framedhslinecorrect\\{-1.5ex}%
   \let\endoflinesave=\\
   \let\\=\@normalcr
   \(\pboxed}%
  {\endpboxed\)%
   \framedhslinecorrect\endoflinesave{.5ex}\hline
   \endtabular
   \parskip=\belowdisplayskip\par\noindent
   \ignorespacesafterend}

\newcommand{\framedhslinecorrect}[2]%
  {#1[#2]}

\newcommand{\framedhs}{\sethscode{framedhscode}}

% The inlinehscode environment is an experimental environment
% that can be used to typeset displayed code inline.

\newenvironment{inlinehscode}%
  {\(\def\column##1##2{}%
   \let\>\undefined\let\<\undefined\let\\\undefined
   \newcommand\>[1][]{}\newcommand\<[1][]{}\newcommand\\[1][]{}%
   \def\fromto##1##2##3{##3}%
   \def\nextline{}}{\) }%

\newcommand{\inlinehs}{\sethscode{inlinehscode}}

% The joincode environment is a separate environment that
% can be used to surround and thereby connect multiple code
% blocks.

\newenvironment{joincode}%
  {\let\orighscode=\hscode
   \let\origendhscode=\endhscode
   \def\endhscode{\def\hscode{\endgroup\def\@currenvir{hscode}\\}\begingroup}
   %\let\SaveRestoreHook=\empty
   %\let\ColumnHook=\empty
   %\let\resethooks=\empty
   \orighscode\def\hscode{\endgroup\def\@currenvir{hscode}}}%
  {\origendhscode
   \global\let\hscode=\orighscode
   \global\let\endhscode=\origendhscode}%

\makeatother
\EndFmtInput
%
\DeclareMathAlphabet{\mathkw}{OT1}{cmss}{bx}{n}


\newcommand{\F}{\mathsf}

\usepackage{textcomp}
\usepackage{listings}
\usepackage{upquote}
\lstset{
    numbers=left,                
    numberstyle=\scriptsize,
    tabsize=4,
    rulecolor=,
    language=java,
        basicstyle=\scriptsize,
        upquote=true,
        aboveskip={1.5\baselineskip},
        columns=fixed,
        showstringspaces=false,
        extendedchars=true,
        breaklines=true,
        prebreak = \raisebox{0ex}[0ex][0ex]{\ensuremath{\hookleftarrow}},
        frame=single,
        showtabs=false,
        showspaces=false,
        showstringspaces=false,
        identifierstyle=\ttfamily,
        keywordstyle=\ttfamily
}



\begin{document}
\title{CS408 Project Report}
\author{Rokas Labeikis}
\maketitle


\section{Introduction}

Frank is strongly typed, strict functional programming language invented by
Sam Lindley and Conor McBride and it is influenced by Paul Blain Levy’s
call-by-push-value calculus. Featuring  a bidirectional effect type system,
effect polymorphism, and effect handlers. This means that Frank supports
type-checked side-effects which only occur where permitted.
Side-effects are comparable to exceptions which suspend the evaluation of the
expression where they occur and give control to a handler which interprets
the command. However, when command is complete depending on the handler the
system could resume from the point it was suspended. Handlers are very similar
to typical functions but their argument processes can communicate in more
advanced ways. So the idea is to utilize this functionality in the web.
Side-effects might be various events such as mouse actions, http requests etc.
and the handler would be the application in the web page.

So, in this project the main goal is to compile Frank to JavaScript and run it
in the browser. So, for example, user would be able to edit their MyPlace pages
using Frank language. This involves creating Virtual Machine (abstract machine)
which can support Frank structure.

\section{Simple Compiler}
\begin{hscode}\SaveRestoreHook
\column{B}{@{}>{\hspre}l<{\hspost}@{}}%
\column{3}{@{}>{\hspre}l<{\hspost}@{}}%
\column{10}{@{}>{\hspre}l<{\hspost}@{}}%
\column{14}{@{}>{\hspre}l<{\hspost}@{}}%
\column{18}{@{}>{\hspre}l<{\hspost}@{}}%
\column{22}{@{}>{\hspre}l<{\hspost}@{}}%
\column{26}{@{}>{\hspre}l<{\hspost}@{}}%
\column{E}{@{}>{\hspre}l<{\hspost}@{}}%
\>[3]{}\mathkw{instance}\;\mathsf{Monad}\;\mathsf{CodeGen}\;\;\mathkw{where}{}\<[E]%
\\
\>[3]{}\hsindent{7}{}\<[10]%
\>[10]{}\Varid{return}\;\Varid{val}\mathrel{=}\mathsf{MkCodeGen}\mathbin{\$}\lambda \Varid{next}\to ([\mskip1.5mu \mskip1.5mu],\Varid{next},\Varid{val}){}\<[E]%
\\
\>[3]{}\hsindent{7}{}\<[10]%
\>[10]{}\Varid{ag}\bind \Varid{a2bg}\mathrel{=}\mathsf{MkCodeGen}\mathbin{\$}\lambda \Varid{next}\to {}\<[E]%
\\
\>[10]{}\hsindent{4}{}\<[14]%
\>[14]{}\mathkw{case}\;\Varid{codeGen}\;\Varid{ag}\;\Varid{next}\;\mathkw{of}{}\<[E]%
\\
\>[14]{}\hsindent{4}{}\<[18]%
\>[18]{}(\Varid{ac},\Varid{next},\Varid{a})\to \mathkw{case}{}\<[E]%
\\
\>[18]{}\hsindent{4}{}\<[22]%
\>[22]{}\Varid{codeGen}\;(\Varid{a2bg}\;\Varid{a})\;\Varid{next}\;\mathkw{of}{}\<[E]%
\\
\>[22]{}\hsindent{4}{}\<[26]%
\>[26]{}(\Varid{bc},\Varid{next},\Varid{b})\to (\Varid{ac}\plus \Varid{bc},\Varid{next},\Varid{b}){}\<[E]%
\ColumnHook
\end{hscode}\resethooks
\begin{hscode}\SaveRestoreHook
\column{B}{@{}>{\hspre}l<{\hspost}@{}}%
\column{5}{@{}>{\hspre}l<{\hspost}@{}}%
\column{E}{@{}>{\hspre}l<{\hspost}@{}}%
\>[5]{}\Varid{genDef}\mathbin{::}\mathsf{String}\to \mathsf{CodeGen}\;\mathsf{Int}{}\<[E]%
\\
\>[5]{}\Varid{genDef}\;\Varid{code}\mathrel{=}\mathsf{MkCodeGen}\mathbin{\$}\lambda \Varid{next}\to ([\mskip1.5mu (\Varid{next},\Varid{code})\mskip1.5mu],\Varid{next}\mathbin{+}\mathrm{1},\Varid{next}){}\<[E]%
\ColumnHook
\end{hscode}\resethooks
\begin{hscode}\SaveRestoreHook
\column{B}{@{}>{\hspre}l<{\hspost}@{}}%
\column{5}{@{}>{\hspre}l<{\hspost}@{}}%
\column{9}{@{}>{\hspre}l<{\hspost}@{}}%
\column{11}{@{}>{\hspre}l<{\hspost}@{}}%
\column{13}{@{}>{\hspre}l<{\hspost}@{}}%
\column{E}{@{}>{\hspre}l<{\hspost}@{}}%
\>[5]{}\Varid{compile}\mathbin{::}\mathsf{Expr}\to \mathsf{CodeGen}\;\mathsf{Int}{}\<[E]%
\\
\>[5]{}\Varid{compile}\;\Varid{e}\mathrel{=}\Varid{help}\;\text{\tt \char34 s\char34}\;\Varid{e}\;\;\mathkw{where}{}\<[E]%
\\
\>[5]{}\hsindent{4}{}\<[9]%
\>[9]{}\Varid{help}\;\Varid{s}\;(\mathsf{Val}\;\Varid{n})\mathrel{=}\Varid{genDef}\mathbin{\$}{}\<[E]%
\\
\>[9]{}\hsindent{2}{}\<[11]%
\>[11]{}\text{\tt \char34 function(s)\char123 return\char123 stack:\char34}\plus \Varid{s}\plus \text{\tt \char34 ,tag:\char92 \char34 num\char92 \char34 ,~data:\char34}\plus \Varid{show}\;\Varid{n}\plus \text{\tt \char34 \char125 \char125 \char34}{}\<[E]%
\\
\>[5]{}\hsindent{4}{}\<[9]%
\>[9]{}\Varid{help}\;\Varid{s}\;(\Varid{e1}\mathbin{:+:}\Varid{e2})\mathrel{=}\mathkw{do}{}\<[E]%
\\
\>[9]{}\hsindent{4}{}\<[13]%
\>[13]{}\Varid{f2}\leftarrow \Varid{compile}\;\Varid{e2}{}\<[E]%
\\
\>[9]{}\hsindent{4}{}\<[13]%
\>[13]{}\Varid{help}\;(\text{\tt \char34 \char123 prev:\char34}\plus \Varid{s}\plus \text{\tt \char34 ,~tag:\char92 \char34 left\char92 \char34 ,~data:\char34}\plus \Varid{show}\;\Varid{f2}\plus \text{\tt \char34 \char125 \char34})\;\Varid{e1}{}\<[E]%
\ColumnHook
\end{hscode}\resethooks

\section{Simple Abstract Machine}

Basic Stack structure:

\begin{lstlisting}
     Mode: {
        stack: { // rest of stack
            prev: "", // previous stack
            tag: "", // opperation 
            data: "" // pointer to next frame
        }
        tag: "",  // data type, example "num" for number
        data: "" // value
    }
\end{lstlisting}


Example program:

\begin{lstlisting}
    var ProgramFoo = [];

ProgramFoo[0] = function (s) {
    return {
        stack: s,
        tag: "num",
        data: 3
    }
};

\end{lstlisting}

This program will place 3 on top of the stack and halt.




\section{Project Evaluation}

Project will be evaluated by other researchers who have a clear view of how the
software should function. They will test the system and give their feedback and
assessment. Evaluators will expect that the existing body of Frank examples should
work in my implementation of the system the same way they do in other
implementations, as well as client-side programming should become easier, example,
writing parsers for data in web forms. Furthermore, project will use continuous
evaluation technique, so it will be evaluated, for example, every two weeks by at
least one researcher, in order to follow Agile software development practices.  

\section{Progress Log}

JANURAY

January 4th: Looked at Frankjr, Shonky, Vole implementations.

January 5th: Tried to install Frankjr for about 4 hours with no success, constant
errors and crashes.

January 7th: Looked at Vole with a bit more detail.

January 9th: Worked on project specification, plan and presentation. Also, tried
bunch of new ways to install frank (cabal way) with no success.

January 10th: Looked at Vole, in particular Machine.lhs and Compile.lhs, Vole.js.

January 11th: Worked on project specification, plan and presentation.

January 16th: Looked at Shonky stack implementation, tried to output it
on the screen.

January 17th: Implemented simple linked list stack in JS, just as a pratice.

January 18th: Looked again at Shonky Abstract Machine. Understood the order how
input program is parsed.

January 24th: Implemented simple Abstract Machine, which can sum 2 numbers.

January 25th: Machine now works fine with stack larger than 2. It uses array to
store functions now. Looking into how to implement throw, catch and compiler.

January 26th: Started to work on simple compiler written in Haskell.

January 30th: Simple compiler code compiles now. However, I still can't run it
because I need to implement Show. My Haskell knowledge proves to be an issue.
Also, laid the groundwork for latex with lex2tex, so I can start working on final
report.

January 31th: Implemented Throw and Cache for Abstract Machine. Worked on final
report. 

FEBRUARY 

February 01: Worked on latext configurations.

February 02: Implemented early versions of stack saving,restoring and variable
assigment. 

\end{document}
